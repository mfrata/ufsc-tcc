\newcommand{\ASKA}{$\text{ASK}^0$}
\newcommand{\ASKS}{$\text{ASK}^*$}

\chapter{Literature Review}

In this chapter it will be discussed some basic concepts to better understand this work. First, it will be introduced the concept of  recommender systems (RS) and their importance to today's online business. Second, the metric used to evaluate the performance of the RS algorithm: lift. After that, a general discussion of clustering will be presented followed by one technique to visualize clustering: PCA with two principal components.


\section{Recommender systems}

A RS is a software which its main purpose, as the name suggests, is to give suggestions or recommendations to a user based on information about the user itself or the context of the items to recommend. They are classified as part of information filtering systems \cite{RecommendersystemWikipedia}. So, another way to interpret the recommendations is to think as a result of a filter applied on a search space, where only the most relevant information is given back to the user. It is important to emphasize that usually the search space on the RSs today are enormous. Meaning that, in terms of time, it is impossible to go one by one. Hence, the importance of these type of system on the applications today.


% \cite{ricci2011introduction}

Nulla vestibulum lorem mollis, placerat urna ac, pharetra mi. Aliquam magna sapien, porta non mauris nec, euismod fermentum nibh. Donec sollicitudin quam eu dignissim interdum. Vestibulum ut condimentum augue. Pellentesque quis felis a neque fringilla faucibus. Integer blandit nunc quis turpis finibus, a efficitur libero pretium. Nulla ac placerat diam. Ut sit amet justo nec purus tristique euismod. Vestibulum vulputate magna sapien, eget lobortis nulla semper in. Sed egestas mauris semper elit sodales semper. Quisque in lacus vitae felis egestas faucibus. ]


\begin{figure}
       \centering
       \includegraphics[width = \linewidth]{fig/dummy.jpg}
       \caption{description image 1}
       \label{image1}
\end{figure}

Sed eu tempor ante. Cras quis facilisis lorem. Donec luctus tincidunt nunc hendrerit tempor. Nam accumsan id augue vitae dignissim. Morbi vel risus lacinia, sodales neque ut, porttitor arcu. Etiam et metus in tortor molestie posuere. Sed convallis a magna ac convallis. Mauris vel elit et turpis convallis varius. Suspendisse interdum lorem ac aliquet lacinia. Vestibulum at imperdiet lectus. Cras ac porttitor elit. 

\begin{figure}
       \centering
       \includegraphics[width = \linewidth]{fig/dummy.png}
       \caption{description image 2}
       \label{image2}
\end{figure}

Sed eu tempor ante. Cras quis facilisis lorem. Donec luctus tincidunt nunc hendrerit tempor. Nam accumsan id augue vitae dignissim. Morbi vel risus lacinia, sodales neque ut, porttitor arcu. Etiam et metus in tortor molestie posuere. Sed convallis a magna ac convallis. Mauris vel elit et turpis convallis varius. Suspendisse interdum lorem ac aliquet lacinia. Vestibulum at imperdiet lectus. Cras ac porttitor elit.

In semper quam posuere tincidunt feugiat. Nunc pharetra nec nulla id condimentum. Aenean pellentesque posuere gravida. Aliquam euismod mi metus, et vestibulum diam dapibus non. Nullam maximus mauris at venenatis pulvinar. Vestibulum ullamcorper, erat et bibendum consectetur, quam dolor ultricies magna, at elementum dolor dui vel neque. Sed volutpat fermentum justo. 


\subsection{Metrics: Lift}

In semper quam posuere tincidunt feugiat. Nunc pharetra nec nulla id condimentum. Aenean pellentesque posuere gravida. Aliquam euismod mi metus, et vestibulum diam dapibus non. Nullam maximus mauris at venenatis pulvinar. Vestibulum ullamcorper, erat et bibendum consectetur, quam dolor ultricies magna, at elementum dolor dui vel neque. Sed volutpat fermentum justo. 

\section{Clustering}

\subsection{Cluster method 1}

O estimador para a variância de uma população de $N$ amostras é dado por

\begin{equation}
	\mathbf{E}[X] = \int_{-\infty}^{\infty} x \cdot p(x) dx.
\end{equation}

\subsection{Cluster method 2}

A variância é definida como

O estimador para a variância de uma população de $N$ amostras é dado por:
\begin{equation}
\hat{\sigma}^2 = \frac{1}{N-1}\sum_{i=0}^{N-1}(x[i] - m_x)^2 = \frac{1}{N-1}\left(\sum_{i=0}^{N-1}(x[i]^2) - m_x^2\right).
\end{equation}


