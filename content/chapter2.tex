\newcommand{\ASKA}{$\text{ASK}^0$}
\newcommand{\ASKS}{$\text{ASK}^*$}

\chapter{Literature Review}

In this chapter it will be discussed some basic concepts to better understand this thesis work. First, it will be introduced the concept of  recommender systems (RS) and their importance to today's online business. Second, the metric used to evaluate the performance of the RS algorithm: lift. After that, a general discussion of clustering will be presented followed by one technique to visualize the clusters of a dataset: PCA with two principal components.


\section{Recommender systems}

A RS is a software which its main purpose, as the name suggests, is to give suggestions \cite{ricci2011introduction} or recommendations to a user based on information about the user itself or the context of the items to recommend. They are classified as part of information filtering systems \cite{RecommendersystemWikipedia}. So, another way to interpret the recommendations is to think as a result of a filter applied on a search space, where only the most relevant information is given back to the user. It is important to emphasize that usually the search space on the RSs today are enormous. Meaning that, in terms of time, it is impossible to go one by one. Hence, the importance of these type of system on the applications today.

\subsection{Types of Recommender Systems}

There are three main types: the ones based on \textbf{collaborative filtering}, the ones on \textbf{content-based filtering}, and \textbf{hybrid}. 

The first one is \textbf{collaborative filtering}, where the recommendations come mainly from information generated by the user and its interaction with the items. For instance, one type of logic is the \textbf{user-based}, where a item is recommended to the user based on what other similar users like, more known as \textit{people like you, also like X}. Another logic is the \textbf{item-based} one, where the RS acts based on the similarity between items, also known as \textit{if you like X you may like X}. The Figure \ref{fig:simi_dist_expected_real} shows an example of user-based collaborative filtering, where the RS wants to predict what is the preference on headphones of user E. Based on users B and C (they voted similarly to E), we can see that, probably its an \textit{dislike}.

\begin{figure}[h]
   \centering
   \includegraphics[width=6cm]{fig/ch2-colab-filt-user-user.jpg}
   \caption{User-based logic on collaborative filtering. Source: \cite{delawareai}}
   \label{fig:colab-filt-user-user}
\end{figure}

The second one is \textbf{content-based filtering}, where the recommendations are based on the features of the items and more importantly: an \textbf{user profile}. The system needs a input from the user, be it interactively like in a sequential manner or historic data, or in a batch way where the user gives a lot information about itself in one shot. Having the profile, the RS can cross this information with the features of the items to look for items that are similar to the user's profile. It can remember the item-based logic from collaborative filtering, but here we have added the information of the user. One example that illustrate content-based, is an article reader service. Imagine when one sign up to the service, it will ask a lot of question to the new user: "What type of articles you prefer: Sports, Politics, Health?", "What kind of Sports do you like: Soccer, Volleyball, Baseball...?". These questions build up a profile to new user and it will be refined as the users utilizes the service. Therefore, if our hypothetical new user answers "Sports" and "Soccer" to the previous questions, the article service will start recommending articles of soccer. But if this new user only thumbs up Brazilian Soccer articles, the RS will learn that and narrow down the recommendations to Brazilian soccer.

Finally, there is the \textbf{hybrid} type, where the Recommendation System is based on a mixture of both previous types. Combining them can improve the recommendations and at the same time deal with their constraints. Collaborative filtering has the problem of \textbf{cold start}, whenever a new item or user is added, it has no attributes. So it will take some time until its attributes are filled up. This problem leads to another one which is the \textbf{sparsity} on the matrix of attributes. Meaning that, there are a lot of missing values. Content-based filtering can supply this data to the RS. And combining with the accuracy of collaborative filtering the system can achieve very personalized recommendations.


\subsection{Benefits on business}

The RSs are present on a myriad of online services today, bringing great value to them. Streaming platforms of music and videos have RSs on their business to increase user retention and engagement such as: 75\% of what people watch on Netflix come from recommendations \cite{HowretailerscankeepupwithconsumersMcKinsey} or that 70\% of people's time spent on YouTube comes from the recommendation of "the Algorithm" \cite{CES2018YouTubesAIrecommendationsdrive70percentofviewingCNET}. Social medias and reading platforms apply these same concepts on their "feed" for the same reasons of streaming platforms, user retention and engagement. E-commerce, on the other hand, want to increase their revenue by recommending similar products, or products that the "same type of customer purchased" when a potential customer is navigating on their online shop. Amazon is a great example of this: 35\% of the purchases on their online shopping come from recommendations \cite{HowretailerscankeepupwithconsumersMcKinsey}. There is even the use of RSs on online dating services \cite{brozovsky2007recommender}, where the RS improve the experience of the user in the search of potential partner while at the same time increasing the monetization of the service. All of these business make use of this technology, but there is also the ones that gave back to the research of RS.

Netflix  is one of the companies that are references on this type of system. They have a variety of algorithms on their RS. And due to the high user base, they can test their results using A/B testing and feedback from the users \cite{gomez2016netflix}. On 2006 they launched a competition, called the Netflix Prize, where the objective was to improve the accuracy of the recommendations by 10\%. The winner would win US\$ 1.000.000. There were more than 2.000 teams and more than 40.000 submissions on their platform. The Netflix prize was an important event to the RSs in general because it increased the awareness of this technology, and its importance to business, worldwide. 

\subsection{Evaluation}

sdadashdasdasdasdadas

\section{Clustering}

\subsection{Principal component Analysis}

exemple equation
\begin{equation}
	\mathbf{E}[X] = \int_{-\infty}^{\infty} x \cdot p(x) dx.
\end{equation}

example equation
\begin{equation}
\hat{\sigma}^2 = \frac{1}{N-1}\sum_{i=0}^{N-1}(x[i] - m_x)^2 = \frac{1}{N-1}\left(\sum_{i=0}^{N-1}(x[i]^2) - m_x^2\right).
\end{equation}


