\chapter{Introduction} \label{cha:introduction}

% Because of a bug --- Restore original headers:
\fancyhead[ER]{\sffamily\footnotesize{\leftmark}}
\fancyhead[OL]{\sffamily\footnotesize{\rightmark}}

Neoway Business Solution is a big data analytics company with a variety of products to several verticals such as Sales and Marketing, Risk and Compliance, Health and others. One of the products of the Sales and Marketing vertical is the OnTarget, which its main objective is to recommend leads for the user, based on its portfolio of clients. The OnTarget will search for leads in a subset of the whole Brazilian's market space, which is composed by all active companies. The user can narrow down the search space based on a set of filters. The Figure \ref{fig:braz_comps_venn_diagram} shows a Venn diagram that illustrate the subsets of the OnTarget. The "Portfolio" set is composed by the user's clients; The "Market" is where the OnTarget will search for the leads, it can vary from a set, defined by the filters, to all the Brazilian active companies except the user's clients.

\begin{figure}
   \centering
   \includegraphics[width = \linewidth]{fig/int-brazil_comps_venn_diagram.jpg}
   \caption{Vein Diagram explaining the sets involved on the OnTarget product.}
   \label{fig:braz_comps_venn_diagram}
\end{figure}

% TODO: Should I explain the how the benchmark was made here or on the methodology chapter?
Recently the algorithm had an update, and a benchmark was made to compare the two versions. Overall the new version showed an improve in the performance of the generation of leads. However, at same time, with the benchmark some studies presented some strange behavior. The Figure \ref{fig:simi_dist_expected_real} shows two similarity distributions plots from the benchmark. Similarity is how similar a company on the market is with the user's clients, it ranges from 0 to 1. The COLOR curve shows the market distribution and the COLOR one the portfolio distribution.

% TODO not sure if "high density probabilities" is a good synonym for "bump"
The left image is the best case scenario of a study, where all the portfolios' companies are close to similarity 1; The market has two high density probabilities: one close to similarity 0 and other close to similarity one, in others words, some companies have no fit with the portfolio and others are alike the portfolio. This second group is the high quality leads recommended to the user.

The right image is an example of a study with a strange behavior. The portfolio distribution has several high densities probabilities locations along the x-axis leading the algorithm to not generate high quality leads (market companies with similarity close to 1). 

%TODO I think the last phrase of this paragraph can be improved.
One hypothesis of the OnTarget team is that in this study (and others alike) the client has a heterogeneous portfolio, meaning that it can have two or more companies profiles in it. The algorithm tries to optimize for the mean profile of the whole portfolio but, sometimes, this is not the optimal solution for that portfolio.

\begin{figure}
   \centering
   \includegraphics[width = \linewidth]{fig/int-simi_dist_expected_real.jpg} 
   \caption{Similarity distribution from the benchmark. The left shows a best case scenario one and the right one a strange-behavior scenario.}
   \label{fig:simi_dist_expected_real}
\end{figure}

This work is one of the several improvements for the OnTarget product road map. And its objective is to analyze if the overall performance of the high quality leads generation improves by clustering the portfolio before running the recommendation algorithm. 

% TODO include citation to  https://en.wikipedia.org/wiki/Firmographics
In order to preserve interpretability the clustering will not take place with all the features available. It will be used only firmographics \cite{wikipedia_firmographics} variables which are variables related to demographics and geographics. For instance, company size, location, number of employees and others. 

% TODO This phrase is good but i think it should be on the conclusion.
% Is important to notice that the clustering with these type of features has high value for the business, since it can improve the performance of the recommendations with a human-understandable solution. 

\section{Objectives}

\subsection{General Objectives}

Analyze if the outcome of the OnTarget improves by clustering the portfolio with firmographics variables before running the recommendation algorithm.

\subsection{Specific Objectives}

\begin{itemize}
    \item Define the firmographics variables.
    \item Define and implement the clustering strategy.
	\item Define and implement the clustering algorithm.
    \item Run the benchmark for the OnTarget with the clustering.
    \item Analyze the metrics and graphics of the benchmark.
\end{itemize}


\section{Work outline}
% TODO is this really needed?

 in the chapter 2 it will have this
 in the chapter 3 it will have that