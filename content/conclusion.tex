\chapter{Conclusion} 
\label{cha:conclusion}

The idea of this work emerged after the creation of a benchmark due to an update in the On Target, which is a recommender system at Neoway that its main purpose is to generate high quality leads for the customers of the company. With the benchmark the team noticed that some studies had multiple high densities areas in the holdout set distributions leading to the hypothesis of multiple profiles in the portfolios. Based on this scenario we proposed a proof of concept on clustering the data from a study using firmographics variables in order to improve the performance of the On Target.

We started this thesis by discussing the context of recommender systems and their relevance for business today. Following, a introduction of the concept of clustering and principal component analysis, both terms that were applied in the procedures to come.

After that, we introduced some terminology of the OT and its benchmark, alongside the explanation of how the leads are scored through block diagrams. Then, we tackled the clustering procedures. In the choice of the algorithm and number of clusters the team decided to adopt a "manual" approach of solving these problems. The clustering strategy and pairing chosen were \nameClusterStrategyA{} and \nameClusterPairingA{}, respectively. From both of these, two experiments were design to test the clustering: \nameExperimentI{} and \nameExperimentII{}.

We analysed their results seeing that the former had a poor performance of approximately -9\% and the latter had a sightly improvement by 2\%. We also, further investigate their results by examining the similarity distribution plots of the studies of the groups former based on the lift gain. In this examination we saw the correlation between the lift and the similarity plot, and other studies' cases that did not workout as expected. Finally, we repeated experiment \nameExperimentII{} with other algorithms of clustering and saw that none of them surpassed the improvement of the "manual" approach.

Outlined all the steps of this work and its outcomes we conclude that \textbf{it is not worth to pursue with this idea} of clustering data with the firmographics variables before the run of the OT. We judged that the value we would get from the amount of work remaining to do is minimal. For instance, the clustering algorithm is not in an optimal solution yet, when comparing to the result achieved by the manual approach. Hence, more work in testing new algorithms or adjusting parameters of a specific one would be required. All of this for an improvement in performance by 2\%. On top of that, this is the mean lift gain amongst all studies, that is, some studies had a decrease over 10\% in the lift while others had an increase. So there is the possibility of hurting the client with the OT in production. If the result had more consistency, for example, over 10\% mean improvement with, let us say, less then 3 studies with negative results then the further improvement of the PoC - the development of a Minimum Viable Product (MVP) - would be included in the development pipeline of the On Target's team. 

\section{Takeaways}

In spite of the unsatisfactory result in the performance attained, this work brought some insights for the team:
\begin{itemize}
    \item the practical approaches, like clustering the data manually, saved a lot of time and resources from the company. It would take too much from the team to develop a solution that clustered the data automatically without even knowing the value that it could bring. The PoC is an opportunity to quickly test an idea before putting the necessary effort to put it on production;
    \item this was an opportunity to know more about the problem. We notice that some of the studies have distinct clusters on their portfolios while others were the case of a single cluster with outliers companies. This correlates well with the way of how Neoway's users use the OT. Some of them segment their portfolio based on their sales strategy, thus the data is already "clustered" by our client. This variability adds a layer of complexity to the design of the system. Moreover, Neoway address more than 20 verticals in the market. And the statistics of the data varies amongst them, in such way that this is not an easy problem. It is not a simple model that fulfils the requirement of consistent recommendations for all of these verticals. Knowing more about how is the data in this problem empowers the team to deliver better solutions;
    \item it showed to us that sometimes the simplest solutions is the best way to go, as seen in experiment \nameExperimentII{}. This is a crucial factor when we take to account the fact that Neoway is a software company and its products are developed by a team of people. A complex solution hinders the maintainability aspect of the product, since it introduces more states for testing, logging and debugging;
    \item we could test the limits of the OT. The minimum data to a valid run was the beginning of the inspection of how many companies must be in the portfolio and market (and their ratio) for the OT run with reliable results. We inspected some examples where the results were awful when their portfolio size were negligible when comparing to the market;
    \item it enabled us to see that the lift (performance) is not the only relevant metric to evaluate the result of a study. The benchmark already brought up the discussion of new metrics for the OT. This work, reinforced that when we discussed in \ref{ch:worth-ment} about the study that had a positive lift gain while the sets in the similarity distribution shifted to lower values. This result raises the awareness of the team to improve the metrics and the consistency of the solution;
\end{itemize}

%%% OBS %%%
% Usually there is a future work section in the TCCS, however I stated that it is not worth it to continue with this.

% \section{Future work}
% Run in the production environment
% Improve the GMM algorithms (mess with the hyper parameters)
% Redo number of clusters assignment based on the results of the GMM clustering