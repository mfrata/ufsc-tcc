\chapter{Conclusion} \label{cha:conclusion}


points for conclusion

the manual clustering had the best perfomance of the clustering algorithms
    meaning that the PCA plot clusters identified visually make sense
    
In contrast to academic approach, this work had several practical approaches like clustering the data manually. .... it would take tooo much time to come with a solution that clustered the data and we would not e able to know if it was good due to the its nature (non supervised). Also the nuber of cluster is not a solved problem on the academic, we would have to come with heuristics to determine this value.

The poc is an opportunity to quickly test an idea (waves of innovation)

on top of that was the simplest solution this is an important factor when thinking about a software that is on production, more peopl will "mess" with the software, so we have to consider maintanability

This was an oportunity to know more about the problem of the OT through the plots, we sa that some of the studies have distinc/clear cluster on it, others have only one with some outliers. Also some of the OT users already "segment" the portfolio to run, so it makes sense

we see tha it is a dificult problem since there are a several different business scenraio. This diverse set of studies possibilis does not enable for a on-fit-all solution

also this work bought the some insights to the product like, scoring the portfolio made us see that our sampling on the holdset is not acurate enough

it tested the minium set of possible data of the OT

performance is not the only relevant metric for the case, we have to consider also the similarity distribution or even new metrics

conslusion: it is not worth it to proceed with this in the case of improve the OT performance. the gain is marginal in contrast to the work needed it

our initial hypothesis was not proved.....

\section*{Future work}

AS mentioned  in (outlier) this was not runned on the original benchmark project, but on a separeted environment. 

It is worth it to runned on production environmetn to see if the result the same result is achieved.

The GMM showed the best performance amng the non manual cluster algorithms with default paremeters, sme twinking on the hyper parameters could improve the perfornce

 on top of that the gmm showed that some of the numer of clusters assignamets is wrong like in Figure ... a review of the manual 
 
\begin{enumerate}
  \item next point to be worked on 
  \item next point to be worked on 
  \item next point to be worked on 
  \item next point to be worked on 
  \item next point to be worked on 
\end{enumerate}
